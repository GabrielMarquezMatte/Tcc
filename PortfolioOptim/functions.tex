\documentclass[a4paper,10pt]{article}
\usepackage[utf8]{inputenc}
\usepackage{enumitem}
\usepackage[top=30mm, bottom=30mm, left=20mm, right=20mm]{geometry}
\usepackage{hyperref}
\usepackage{amsmath}

\begin{document}

\begin{center}
    {\Large \textbf{Derivação das funções de otimização}}
\end{center}

\section*{Menor risco de portfólio}

\begin{equation*}
    \begin{aligned}
        \min_{w} \sigma_{p}^{2} &= \min_{w} w^{T} \Sigma w \\
        \text{sujeito a} \quad & w^{T} \mathbf{1} = 1
    \end{aligned}
\end{equation*}

Para resolver esse problema, usamos o método dos multiplicadores de Lagrange. Primeiro, definimos a função Lagrangiana:

\begin{equation*}
    \mathcal{L}(w, \lambda) = w^{T} \Sigma w + \lambda (w^{T} \mathbf{1} - 1)
\end{equation*}

Agora, realizamos as derivações parciais, montando o sistema de equações:

\begin{equation*}
    \begin{aligned}
        \frac{\partial \mathcal{L}}{\partial w} &= 2 \Sigma w + \lambda \mathbf{1} = 0 \\
        \frac{\partial \mathcal{L}}{\partial \lambda} &= w^{T} \mathbf{1} - 1 = 0
    \end{aligned}
\end{equation*}

Resultando em um sistema de equações lineares, representado na forma matricial:

\begin{equation*}
    \begin{bmatrix}
        2 \Sigma & \mathbf{1} \\
        \mathbf{1}^{T} & 0
    \end{bmatrix}
    \begin{bmatrix}
        w \\
        \lambda
    \end{bmatrix}
    =
    \begin{bmatrix}
        0 \\
        1
    \end{bmatrix}
\end{equation*}

\section*{Menor risco para um retorno esperado}

\begin{equation*}
    \begin{aligned}
        \min_{w} \sigma_{p}^{2} &= \min_{w} w^{T} \Sigma w \\
        \text{sujeito a} \quad & w^{T} \mathbf{1} = 1 \\
        & w^{T} \mu = \mu_{p}
    \end{aligned}
\end{equation*}

Para resolver esse problema, usamos o método dos multiplicadores de Lagrange. Primeiro, definimos a função Lagrangiana:

\begin{equation*}
    \mathcal{L}(w, \lambda, \gamma) = w^{T} \Sigma w + \lambda (w^{T} \mathbf{1} - 1) + \gamma (w^{T} \mu - \mu_{p})
\end{equation*}

Agora, realizamos as derivações parciais, montando o sistema de equações:

\begin{equation*}
    \begin{aligned}
        \frac{\partial \mathcal{L}}{\partial w} &= 2 \Sigma w + \lambda \mathbf{1} + \gamma \mu = 0 \\
        \frac{\partial \mathcal{L}}{\partial \lambda} &= w^{T} \mathbf{1} - 1 = 0 \\
        \frac{\partial \mathcal{L}}{\partial \gamma} &= w^{T} \mu - \mu_{p} = 0
    \end{aligned}
\end{equation*}

Resultando em um sistema de equações lineares, representado na forma matricial:

\begin{equation*}
    \begin{bmatrix}
        2 \Sigma & \mathbf{1} & \mu \\
        \mathbf{1}^{T} & 0 & 0 \\
        \mu^{T} & 0 & 0
    \end{bmatrix}
    \begin{bmatrix}
        w \\
        \lambda \\
        \gamma
    \end{bmatrix}
    =
    \begin{bmatrix}
        0 \\
        1 \\
        \mu_{p}
    \end{bmatrix}
\end{equation*}

\section*{Maior utilidade esperada}

\begin{equation*}
    \begin{aligned}
        \max_{w} U &= \max_{w} w^{T} \mu - \frac{\gamma}{2} w^{T} \Sigma w \\
        \text{sujeito a} \quad & w^{T} \mathbf{1} = 1
    \end{aligned}
\end{equation*}

Para resolver esse problema, usamos o método dos multiplicadores de Lagrange. Primeiro, definimos a função Lagrangiana:

\begin{equation*}
    \mathcal{L}(w, \lambda) = w^{T} \mu - \frac{\gamma}{2} w^{T} \Sigma w + \lambda (w^{T} \mathbf{1} - 1)
\end{equation*}

Agora, realizamos as derivações parciais, montando o sistema de equações:

\begin{equation*}
    \begin{aligned}
        \frac{\partial \mathcal{L}}{\partial w} &= \mu - \gamma \Sigma w + \lambda \mathbf{1} = 0 \\
        \frac{\partial \mathcal{L}}{\partial \lambda} &= w^{T} \mathbf{1} - 1 = 0
    \end{aligned}
\end{equation*}

Resultando em um sistema de equações lineares, representado na forma matricial:

\begin{equation*}
    \begin{bmatrix}
        \Sigma & \mathbf{1} \\
        \mathbf{1}^{T} & 0
    \end{bmatrix}
    \begin{bmatrix}
        w \\
        \lambda
    \end{bmatrix}
    =
    \begin{bmatrix}
        \mu \\
        1
    \end{bmatrix}
\end{equation*}

\section*{Maior Sharpe Ratio}

\begin{equation*}
    \begin{aligned}
        \max_{w} SR &= \max_{w} \frac{w^{T} \mu - r_{f}}{\sqrt{w^{T} \Sigma w}} \\
        \text{sujeito a} \quad & w^{T} \mathbf{1} = 1
    \end{aligned}
\end{equation*}

A dificuldade desse problema se dá ao fato da função objetivo ser uma razão (não convexa).

\end{document}